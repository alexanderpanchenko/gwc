\documentclass{beamer}

\usetheme{uhh}
\showtotalframenumber
\showuhhlogoeachframe
\showsections

\usepackage{amsmath}
\usepackage{graphicx}
\usepackage{color}
\DeclareMathOperator*{\argmin}{arg\,min}

\usepackage{listings}
\lstset{
  language=python
  }


\title{Inducing Interpretable Word Senses for WSD and Enrichment of Lexical Resources}

\institute[University of Hamburg] 
{
  Faculty of Mathematics, Informatics, and Natural Sciences \\
  Department of Informatics\\
  Language Technology Group
}


\author{Alexander Panchenko \\University of Hamburg, Germany}
\date[11.02.2018]{Jan 11, 2018}


\AtBeginSection[]
{
   %%%%% section title
   % This is how it would look like in Beamer:
   % \begin{frame}
   %     \frametitle{Overview}
   %     \tableofcontents[sections={2-3},currentsection,sectionstyle=show/hide,subsectionstyle=hide]
   % \end{frame}
  \begin{frame}[plain]
  \begin{tikzpicture}[overlay]
    \relax%
    \fill[blueuhh,opacity=1] (-10,-10)
    rectangle(\the\paperwidth,\the\paperheight);
  \end{tikzpicture}
   \begin{tikzpicture}[overlay]
    \relax%
    \fill[white,opacity=1] (-5,-1.2)
    rectangle(\the\paperwidth,0.5) node[pos=0.5,black]{\LARGE\insertsectionhead};
  \end{tikzpicture}
  \end{frame}

  %%%% add subsection to show navigation dots
  \subsection{}
}

\begin{document}

\maketitle

%\begin{frame}
%  \frametitle{Overview}
%
%  \begin{itemize}
%		\item Tensorflow Introduction / First session
%		\item Regression models
%		\item Neural tagger (DNN)
%		\item Implementing Word2Vec	
%		\item \textcolor{greyuhh}{Introduction to Tensorboard}
%		\item\textcolor{greyuhh}{Neural tagger (LSTM)}
%		\item \textcolor{greyuhh}{RNN language model (LSTM)}
%		\item \textcolor{greyuhh}{Maybe: Convolutions}
%	\end{itemize}
%\end{frame}

\section{Motivation}

\begin{frame}
\frametitle{Motivation: Speech recognition}

\vspace{5mm}


\end{frame}


\section{Conclusion}

\begin{frame}{Summary}

\begin{itemize}
	\item How to \textbf{induce word senses} and \textbf{semantic classes} from text and synonyms.
    \vspace{1em}
    \pause
    
	\item \textbf{Interpretability can be added} on the top of induced word senses in a model agnostic way. 
	\vspace{1em}
    \pause
	
	\item Hypernymy labels \textbf{improve hypernymy extraction}. 
	\vspace{1em}
	\pause
	
	\item Linking induced word senses to lexical resources:
	\begin{itemize} 
		\item improves \textbf{performance of WSD};
		\item can be used to \textbf{enrich lexical resources} with new senses.
	\end{itemize}
	
	
\end{itemize}


\end{frame}


\begin{frame}{A New Shared Task on WSI\&D}
  
  \begin{itemize}
  \item Participate in an ACL SIGSLAV sponsored shared task on \textbf{word sense induction and disambiguation} for Russian!
  
  
 \end{itemize} 
  
  \begin{block}{A lexical sample task evaluated using the ARI measure }
  \begin{itemize}
  	\item Target word, e.g. ``bank'' (in Russian).
  	\item Contexts where the word occurs.
  	\item You need to group the contexts by senses.
  \end{itemize}
  %You are given a word, e.g. bank and contexts where the word occurs. %e.g. bank is a financial institution that accepts deposits and river bank is a slope beside a body of water.
  % You need to cluster these contexts in the (unknown in advance) number of clusters which correspond to senses of the word. %In this example, you want to have two groups with the contexts of the company and the area senses of the word bank.
  \end{block}
  
  \pause
  \begin{itemize}
    \item \textbf{More details}: \url{http://russe.nlpub.org/2018/wsi}
  \item You can participate by \textbf{31.01.2018}.
  %\item The submission is via the CodaLab: you can see an \textbf{estimate of your rank immediately}.
    
  \end{itemize}
  
\end{frame}

\begin{frame}{}
\Huge{Thank you!}
\end{frame}

\end{document}

